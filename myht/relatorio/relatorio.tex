\documentclass[a4paper, 11pt]{article}
\usepackage[top=3cm, bottom=3cm, left = 2cm, right = 2cm]{geometry}
\usepackage[brazilian]{babel}
\usepackage{setspace}
\usepackage{graphicx}

\title{Relatório: Trabalho 2 - Cuckoo Hashing}
\author{Pedro Folloni Pesserl\\
\textit{Departamento de Informática}\\
\textit{Universidade Federal do Paraná - UFPR}\\
Curitiba, Brasil\\
\texttt{pfp22@inf.ufpr.br}}
\date{}

\begin{document}
\maketitle

\begin{abstract}
\begin{singlespace}
Este relatório documenta o software desenvolvido para o segundo trabalho prático da
disciplina CI1057 - Algoritmos e Estruturas de Dados 3, do Departamento de Informática
da UFPR. O programa é uma implementação em linguagem C da busca, inclusão e exclusão
de valores em uma tabela hash de endereçamento aberto, simulando o algoritmo de
Cuckoo Hash.
\end{singlespace}
\end{abstract}

\section{Estruturas de dados utilizadas}
A tabela hash foi implementada usando as seguintes estruturas, definidas na biblioteca
\texttt{libchash.h}:
\begin{itemize}
    \item Entrada da tabela: contém uma chave de busca, uma variável que especifica se
        a entrada está vazia ou não, e uma que especifica se a chave foi excluída.
    \begin{verbatim}
struct Entrada {
    int chave, vazia, delet;
};
    \end{verbatim}

    \item Tabela Hash: contém dois vetores de Entradas, um para cada tabela do Cuckoo
        Hash (OBS: \texttt{TABLESIZE} é um macro, definido nessa implementação como 11).
    \begin{verbatim}
struct Hash {
    struct Entrada t1[TABLESIZE], t2[TABLESIZE];
};
    \end{verbatim}
\end{itemize}

\section{Bibliotecas desenvolvidas}
A biblioteca desenvolvida para o trabalho foi a \texttt{libchash.[hc]}, que define as
structs acima, além das seguintes funções (considere, para as funções a seguir,
$m = \texttt{TABLESIZE}$):
\begin{itemize}
    \item \texttt{struct Hash cria\_tabela();} -- Cria uma tabela hash e marca todas as
        entradas como vazias.
    \item \texttt{int h1(int k);} -- Função hash para a tabela 1, dada por
        $h_1(k) = k \bmod m$.
    \item \texttt{int h2(int k);} -- Função hash para a tabela 2, dada por 
        $h_2(k) = \lfloor m(0,9k - \lfloor 0,9k \rfloor) \rfloor$.
    \item \texttt{int busca\_chash(struct Hash *hash, int k);} -- Busca uma chave na
        tabela. Se a posição $h_1(\texttt{k})$ na tabela 1 estiver vazia, checa a
        posição $h_2(\texttt{k})$ da tabela 2 (veja detalhes sobre esse comportamento
        na seção 3, adiante). Se a chave não estiver em nenhuma das tabelas, retorna
        -1. Se estiver na tabela 1, retorna a posição (um inteiro $\texttt{i} \in
        [0..10]$). Se estiver na tabela 2, retorna a posição mais $m$ (um inteiro
        $\texttt{j} \in [11..21]$). Esses valores foram escolhidos para facilitar a
        interpretação do resultado em um possível uso da função de busca.
    \item \texttt{int insere\_chash(struct Hash *hash, int k);} -- Insere uma chave na
        tabela. Se a inserção ocorrer normalmente, retorna 0. Se houver tentativa de
        inserir um valor duplicado, não faz nada e retorna 1. Se o valor não puder ser
        inserido (colisão na tabela 2), não faz nada e retorna 2.
    \item \texttt{int exclui\_chash(struct Hash *hash, int k);} -- Exclui um valor da
        tabela hash, seja da tabela 1 ou da tabela 2. Na prática, a posição na tabela é
        marcada como vazia. Se a exclusão ocorrer normalmente, retorna 0. Se a chave
        \texttt{k} não estiver na tabela, não faz nada e retorna 1.
    \item \texttt{void imprime\_chash(struct Hash *hash);} -- Imprime as chaves
        armazenadas na tabela em ordem crescente, juntamente com a tabela na qual a
        chave \texttt{k} se encontra (\texttt{T1} ou \texttt{T2}), seguido pela posição
        da chave na tabela (inteiro $\texttt{i} \in [0..10]$). A saída segue o formato
        \texttt{k,T[1|2],i}.

        OBS: Para realizar a impressão em ordem, foi usada a função qsort, da biblioteca
        padrão da linguagem C. Para tanto, foi desenvolvida uma nova função, definida
        com a palavra-chave \texttt{static} (ou seja, que só pode ser acessada pelo
        código da \texttt{libchash.c}):
    \item \texttt{static int compara\_chaves(const void *a, const void *b);} -- Compara
        duas estruturas \texttt{Entrada}, por meio do campo \texttt{chave}: retorna um
        valor menor que, igual a, ou maior que 0 se a primeira chave for, respectivamente,
        menor que, igual à, ou maior que a segunda.
\end{itemize}


\end{document}
