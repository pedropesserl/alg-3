\documentclass[a4paper, 11pt]{article}
\usepackage[top=3cm, bottom=3cm, left = 2cm, right = 2cm]{geometry}
\usepackage[brazilian]{babel}
\usepackage{setspace}
\usepackage{graphicx}

\title{Relatório: Trabalho 2 - Cuckoo Hashing}
\author{Pedro Folloni Pesserl\\
\textit{Departamento de Informática}\\
\textit{Universidade Federal do Paraná - UFPR}\\
Curitiba, Brasil\\
\texttt{pfp22@inf.ufpr.br}}
\date{}

\begin{document}
\maketitle

\begin{abstract}
\begin{singlespace}
Este relatório documenta o software desenvolvido para o segundo trabalho prático da
disciplina CI1057 - Algoritmos e Estruturas de Dados 3, do Departamento de Informática
da UFPR. O programa é uma implementação em linguagem C da busca, inclusão e exclusão
de valores em uma tabela hash de endereçamento aberto, simulando o algoritmo de
Cuckoo Hash.
\end{singlespace}
\end{abstract}

\section{Estruturas de dados utilizadas}
A tabela hash foi implementada usando as seguintes estruturas, definidas na biblioteca
\texttt{libchash.h}:
\begin{itemize}
    \item Entrada da tabela: contém uma chave de busca e uma variável que especifica se
        a entrada está vazia ou não.
    \begin{verbatim}
struct Entrada {
    int chave, vazia;
};
    \end{verbatim}

    \item Tabela Hash: contém dois vetores de Entradas, um para cada tabela do Cuckoo
        Hash (OBS: \texttt{TABLESIZE} é um macro, definido nessa implementação como 11).
    \begin{verbatim}
struct Hash {
    struct Entrada t1[TABLESIZE], t2[TABLESIZE];
};
    \end{verbatim}
\end{itemize}

\section{Bibliotecas desenvolvidas}
A biblioteca desenvolvida para o trabalho foi a \texttt{libchash.[hc]}, que define as
structs acima, além das seguintes funções (considere, para as funções a seguir,
$m = \texttt{TABLESIZE}$):
\begin{itemize}
    \item \texttt{struct Hash cria\_tabela();} -- Cria uma tabela hash e marca todas as
        entradas como vazias.
    \item \texttt{int h1(int k);} -- Função hash para a tabela 1, dada por
        $h_1(k) = k \bmod m$.
    \item \texttt{int h2(int k);} -- Função hash para a tabela 2, dada por 
        $h_2(k) = \lfloor m(0,9k - \lfloor 0,9k \rfloor) \rfloor$.
    \item \texttt{int busca\_chash(struct Hash *hash, int k);} -- Busca uma chave na
        tabela. Se a posição $h_1(\texttt{k})$ na tabela 1 estiver vazia, checa a
        posição $h_2(\texttt{k})$ da tabela 2 (veja detalhes sobre esse comportamento
        na seção 3, adiante). Se a chave não estiver em nenhuma das tabelas, retorna
        -1. Se estiver na tabela 1, retorna a posição (um inteiro $\texttt{i} \in
        [0..10]$). Se estiver na tabela 2, retorna a posição mais $m$ (um inteiro
        $\texttt{j} \in [11..21]$). Esses valores foram escolhidos para facilitar a
        interpretação do resultado em um possível uso da função de busca.
    \item \texttt{int insere\_chash(struct Hash *hash, int k);} -- Insere uma chave na
        tabela. Se a inserção ocorrer normalmente, retorna 0. Se houver tentativa de
        inserir um valor duplicado, não faz nada e retorna 1. Se o valor não puder ser
        inserido (colisão na tabela 2), não faz nada e retorna 2.
    \item \texttt{int exclui\_chash(struct Hash *hash, int k);} -- Exclui um valor da
        tabela hash, seja da tabela 1 ou da tabela 2. Na prática, a posição na tabela é
        marcada como vazia. Se a exclusão ocorrer normalmente, retorna 0. Se a chave
        \texttt{k} não estiver na tabela, não faz nada e retorna 1.
    \item \texttt{void imprime\_chash(struct Hash *hash);} -- Imprime as chaves
        armazenadas na tabela em ordem crescente, juntamente com a tabela na qual a
        chave \texttt{k} se encontra (\texttt{T1} ou \texttt{T2}), seguido pela posição
        da chave na tabela (inteiro $\texttt{i} \in [0..10]$). A saída segue o formato
        \texttt{k,T[1|2],i}.

        OBS: Para realizar a impressão em ordem, foi usada a função qsort, da biblioteca
        padrão da linguagem C. Para tanto, foi desenvolvida uma nova estrutura,
        \texttt{Entrada\_id}:
        \begin{verbatim}
struct Entrada_id {
    int chave, vazia, tabela, pos;
};
        \end{verbatim}
        Dessa forma, é criado um vetor com as entradas das duas tabelas, e esse
        vetor é ordenado pelas chaves das estruturas. Depois, as entradas não vazias
        são impressas seguindo o formato acima.
\end{itemize}

\section{Decisão pelos campos definidos na estrutura \texttt{Entrada}}
Inicialmente, a estrutura \texttt{Entrada} continha três campos: \texttt{chave},
\texttt{vazia} e \texttt{excluida}. Esse último servia para determinar se uma entrada
havia sido excluída após ser inserida. Isso porque, na função de busca, se a posição
de uma chave na tabela 1 estivesse vazia, a função retornava -1 imediatamente, mas
pode ser necessário checar a tabela 2 também.

Porém, esse campo só era útil na tabela 1, e seria inutilizado na tabela 2; além disso,
essa característica só faz diferença no comportamento da função de busca considerando
as primeiras vezes que a tabela é utilizada: após várias inserções e remoções, todas
as posições da tabela 1 serão marcadas como não-vazias, e a função checará a tabela 2
de qualquer maneira.

Então, em lugar de criar duas estruturas diferentes (uma para a tabela 1, com os três
campos, e outra para a tabela 2, com apenas dois), e considerando que a função de busca
não perde grande desempenho checando as duas tabelas em toda execução, foi decidido
usar apenas os campos \texttt{chave} e \texttt{vazia}. Isso torna a implementação mais
simples e ajuda na legibilidade do código.

\end{document}
